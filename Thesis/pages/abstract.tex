\chapter{\abstractname}

Simulation is a useful technique for Hardware Software Co-development. It is performed at various levels of abstraction to serve different purposes. Instruction Set Simulation is the lowest level of abstraction where the processor pipeline is simulated in detail, to allow hardware developers to test their modifications and evaluate the impact on performance. At higher levels of abstraction, simulation provides developers with a tangible environment for early software development. The focus of this project is on simulation for performance estimation, namely, estimation of time and power consumed in running a benchmark application on a target processor.

While Instruction Set Simulators are known to be highly accurate, they are difficult to develop and slow to execute because of the level of detail they address. Host Compiled Simulation is a technique to accelerate performance estimation with negligible impact on accuracy. The idea is to instrument\footnote{Instrumentation is a technique to modify the source code of an application in order to collect statistics at run-time. This may be used to measure performance of the application, or diagnose errors.} the source code, by taking into consideration the behaviour of the target processor. The instrumented source code is compiled and run on the Host Machine. The technique relies on the assumption that performance of each basic block\footnote{A basic block in a program is a series of instructions which are executed sequentially. The basic block does not contain branch instructions.} in the binary code can be accurately estimated on a certain processor by emulating the pipeline. Other aspects that affect performance, like resources spent in memory access can be accounted for, and a fairly accurate estimate of the time and power consumed can be estimated. 

In this project, a tool to perform Host Compiled Simulation was developed. This thesis discusses the state-of-art in simulation. It explains the approach used to develop this tool. The results showing accuracy of estimations from this approach are presented.
