\chapter{\abstractname}

Simulation is a useful technique for Hardware Software Co-development. It is used by architects to analyse the performance impacts of design decisions at early stage in hardware development. The popular approach is called \gls{cas}. The micro-architecture of the processor is simulated in great detail. This gives accurate estimation of performance of the system. However, with growing complexity of modern processors developing \gls{cas} is very difficult. Long running software scenarios are used to analyse system bottlenecks. \gls{cas} is not suited for such use-case due to slow speed of execution.

\gls{hcs} is a technique to speed up benchmarking of processors. At a minimal cost of accuracy, \gls{hcs} runs much faster compared to \gls{cas}. It is easier to understand and develop. \gls{hcs} is based on the technique of instrumentation. Source code of a benchmarking application is analysed and instrumented to extract trace information which will be used to estimate the time spent in execution on the target processor\footnote{The term target processor is used to refer to the processor that is being simulated.}. The instrumented code is run on the Host Machine\footnote{The term host machine is used to refer to the computer where the simulation will be run.}. In this thesis, the technique to automate the instrumentation of source code has been described. The balance between performance and power consumption is critical and delicate. The approach has been extended to provide estimates of power consumption. This approach can speed up iterations in the Design Space Exploration Phase, and reduce the cost and effort.

Results from various benchmark suites show 98\% and higher accuracy of estimates with average speeds of execution of 2000 MIPS. 

