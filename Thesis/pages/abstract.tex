\chapter{\abstractname}

Simulation or Virtual Prototyping is a widely used technique for Hardware/Software Co-development. A model of the target processor\footnote{The term target processor is used to refer to the processor that is being simulated.} is created which represents the key characteristics and behaviour. It is used by hardware architects to analyse the performance impacts of design decisions at early stage in Embedded System development. Software Engineers use it as a tangible environment for early software development, and to analyse and optimize software performance. Simulation helps reduce critical issues, which would otherwise become visible only in the Integration Phase.

The popular approach is called \gls{iss}. Each instruction in the binary code is read, and the model is operated according to the instruction. For highly accurate timing estimation, the micro-architecture of the processor is modelled in great detail. This is called \gls{cas}. With ever increasing complexity of processors, developing and maintaining \gls{cas} is tedious and time-consuming. \gls{cas} is not suited for simulating long-running software scenarios due to the inhibiting slow speeds of execution. \gls{iss} can be performed at higher levels of abstraction to increase the execution speed at the cost of accuracy of estimation. However, since each instruction must be read and executed, the speed of execution is much slower compared to native execution speed of the target processor.

Researchers have focussed on developing better techniques for benchmarking embedded systems. The focus is on reducing complexity, improving execution speed and providing sufficiently accurate estimates of performance. Such a technique will help architects in identifying critical issues earlier and leaving more time for engineers to iterate and explore the design space.

\gls{hcs} is a popular reserach topic in this area. \gls{hcs} is based on the technique of instrumentation. Source code of an application is analysed and instrumented to extract trace information which will be used to estimate the time spent in execution on the target processor. The instrumented code is run on the Host Machine\footnote{The term host machine is used to refer to the computer where the simulation will be run.}. 

In this work, the technique to automate the instrumentation of source code has been developed. The approach has been extended to provide estimates of power consumption. This approach can speed up iterations in the Design Space Exploration Phase, and reduce the cost and effort. Results from various benchmark suites show 98\% and higher accuracy of estimates with average speeds of execution of 2000 MIPS. 

