\chapter{Source Code to ISC Conversion}
\label{app:ir2cConversion}

The approach to convert Source Code to Intermediate Source Code has been inspired from \cite{RBA2013}. The Retargettable Back-Annotator (RBA) is made available by \cite{RBA2013} under the new BSD license. This tool performs instrumentation, and is customized for the PowerPC architecture. The python code to translate Intermediate Code in GIMPLE to ISC has been reused from this tool.

The motivation behind this approach was covered in detail in Section [TODO]. In this section, some brief examples of the ISC code are shown to familiarize the reader with ISC, and illustrate the benefit of converting Source Code to ISC.

Example \ref{ex:memcpySrc} shows a snippet of code from sha benchmark application from the WCET suite \cite{WCET}. This is a clone of the standard \emph{memcpy} function, and has been chosen for this illustration because of its simplicity.

\begin{Example}[h]
\begin{lstlisting}
void* my_memcpy(void* dest, const void* src, size_t count) {
  char* dst8 = (char*)dest;
  char* src8 = (char*)src;

  while (count--) {
      *dst8++ = *src8++;
  }

  return dest;
}
\end{lstlisting}
\caption{Source Code of a function to copy memory content}
\label{ex:memcpySrc}
\end{Example}

Example \ref{ex:memcpyObj} shows the Intermediate Source Code generated by the tool. It can easily be verified, that the ISC has same functionality as the source code.

\begin{Example}[h]
\begin{lstlisting}
void* my_memcpy(void* dest, const void* src, size_t count) {
  uintptr_t ivtmp_23;

my_memcpybb_2:
//  # PRED: ENTRY [100.0%]  (fallthru,exec)
  if (count != 0)
    goto my_memcpybb_5;
  else
    goto my_memcpybb_4;
//  # SUCC: 5 [91.0%]  (true,exec) 4 [9.0%]  (false,exec)

my_memcpybb_5:
//  # PRED: 2 [91.0%]  (true,exec)
  ivtmp_23 = 0;
//  # SUCC: 3 [100.0%]  (fallthru)

my_memcpybb_3:
//  # PRED: 3 [91.0%]  (true,exec) 5 [100.0%]  (fallthru)
  *(char *)((uintptr_t)dest + (uintptr_t)ivtmp_23) = *(char *)((uintptr_t)src + (uintptr_t)ivtmp_23);
  ivtmp_23 = ivtmp_23 + 1;
  if (ivtmp_23 != count)
    goto my_memcpybb_3;
  else
    goto my_memcpybb_4;
//  # SUCC: 3 [91.0%]  (true,exec) 4 [9.0%]  (false,exec)

my_memcpybb_4:
//  # PRED: 3 [9.0%]  (false,exec) 2 [9.0%]  (false,exec)
  return (uintptr_t)dest;
//  # SUCC: EXIT [100.0%] 

}
\end{lstlisting}
\caption{Intermediate Source Code generated from Example \ref{ex:memcpySrc}}
\label{ex:memcpyObj}
\end{Example}

Note, that in the ISC the basic blocks are already marked by labels. The \emph{while} loop in the source code has been implemented as an \emph{if-else} construct and \emph{goto} statements in the block labelled \emph{my\_memcpybb\_3}. The Control Flow graph is easy to extract from the Instrumented Source Code.
