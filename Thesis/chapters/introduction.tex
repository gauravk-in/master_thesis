\chapter{Introduction}\label{chapter:introduction}

\section{Simulation}

Simulation is the technique to imitate the behaviour of a system. It is generally used when developing or working with the real system is expensive or difficult. 

Simulation is widely used in Hardware Software Co-development. The behaviour of a processor is simulated. It allows Hardware Developers to analyse and validate the performance impacts of design decisions at early stage of hardware development. This allows them to save the crucial effort and cost involved in fabricating hardware.

The widely used approach in this area, is called \gls{iss}. In \gls{iss} the processor micro-architecture is simulated in great detail. Each stage of processor pipeline is simulated along with other building blocks of the processor like Caching and Branch Prediction Units. This approach provides cycle accurate estimates of performance. However, \gls{iss} is difficult to develop and slow to execute because of the amount of details that are simulated. \gls{iss} is not suitable for simulation of long running software scenarios.

The focus of this research is to enable fast simulation of single core, embedded processors to provide accurate estimation of time and power spent in executing benchmark applications.

\section{Related Work}
Considerable work has been done in this area of research. The techniques developed can be roughly divided into two approaches. 

\subsection{Sampling Based Approach}
Sampling is an approach used in statistical analysis. Small, yet representative samples are chosen from a vast amount of data. These samples are analysed in detail, and the results are interpolated to gather information about the entire data set.

In Sampling Based Approach, certain samples of the execution trace are simulated in detail using Instruction Set Simulation. The rest of the execution is carried out using Functional Simulation at a higher level of abstraction. This leads to a speed up in simulation.

Research in this area mainly focuses on developing procedures to select the samples.

% TODO: Add content

\subsection{Host Compiled Simulation}
Host Compiled Simulation is based on approach of \gls{sci}. \gls{sci} is the process of modifying source code to collect performance statistics and generate trace information during run-time. 

To accurately estimate the performance of a processor, we must take into consideration the effects due to following optimization techniques used in modern processors.
\begin{itemize} \itemsep -8pt
\item Processor Pipelining
\item Branch Prediction
\item Memory Caching and Prefetching
\end{itemize}
The instrumentation enables simulation of these

\section{Focus}

\subsection{Simple Example}

\section{Thesis Outline}